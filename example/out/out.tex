\documentclass{article}
\usepackage{hyperref}

\hypersetup{
    colorlinks=true,
    linkcolor=blue,
    filecolor=magenta,
    urlcolor=cyan,
    pdftitle={Overleaf Example},
    pdfpagemode=FullScreen,
}

\usepackage{amsmath}
\usepackage{float}
\usepackage{wrapfig}
\usepackage{caption}
\usepackage{subcaption}

\title{Hi}
\author{Tyger375}
\date{}
\begin{document}

\maketitle
\tableofcontents
\begin{center}
TEST COMPONENTS
test

\end{center}

CIAO test ok

\paragraph{TEST}\label{hello}

\section{test}\label{sec:hello}

\subsection{test3}\label{sec:test}

Hi how are you? I'm fine\\
ok, \textbf{let's gooo}\\
\textit{heheheha}
\href{https://youtube.com}{a youtube link}\ref{sec:hello}\underline{hello} how are you? ciaoc
\begin{center}
Hi \

\end{center}
$$ciao$$
\section{test2}

\newpage
heheheha
\begin{enumerate}
\item Test \
\

\end{enumerate}
\begin{equation} 
\\
\sqrt{x^2-1}
\times 2
\binom{\sqrt{x-2}}{test}
\frac{\sqrt{x-2}}{test}\end{equation}
\begin{wrapfigure}[3\textwidth]{l}{5\textwidth}

\end{wrapfigure}
\begin{figure}[H]
\begin{subfigure}[H]{2\textwidth}

\end{subfigure}

\end{figure}
Ciaoaoo
\begin{equation} 
v^{AB}
_{n}
= 2\\
v
\textsuperscript{AB}
\textsubscript{n}
= 2\\
\int_{i = 0}^{n}
\sum_{i = 1}^{\infty}\end{equation}
MATH:\\
\begin{equation} \label{test}
\begin{split}
C & = n \cdot n\\
& = 2
\end{split}\end{equation}
\begin{equation} 
C = n \cdot n = 2\end{equation}
EQUATIONS:\\
\begin{align*} 
A + B &= 2\\
B &= 3\\
A &= -1\end{align*}
\begin{equation} 
\lim_{h \rightarrow 0
}
2x + h\end{equation}
\textit{m\textsuperscript{AB} and}\\
\textit{hi how are you?}\\
Where the two \textit{r} is the orthogonal of the direction \textbf{AP} and \textbf{BP}

\end{document}
